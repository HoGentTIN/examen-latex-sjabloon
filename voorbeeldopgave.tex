\documentclass{hogent-exam}

%---------- Info over het examen ----------------------------------------------

\ExamDate{23 januari 2020}
\AcademicYear{2019-2020}
\StudyProgramme{Toegepaste informatica}
\MainSubject{}                           % Afstudeerrichting
\Year{2}                                 % Jaar in de opleiding (Modeltraject)
\CourseUnit{Besturingssystemen: Linux}
\Version{Reeks 1}                        % vb. Reeks 1, Inhaalexamen
\Instructors{Bert Van Vreckem}

% Voorbeeldoplossing? (ja -> \solutiontrue; nee -> \solutionfalse)
\solutionfalse

\begin{document}

\maketitle

%---------- Hulpmiddelen ------------------------------------------------------
% Toegelaten hulpmiddelen. Laat leeg als er geen hulpmiddelen gebruikt mogen
% worden.
\Supports{}

% Voorbeeld:
% \Supports{%
%   \begin{itemize}
%     \item Rekenmachine
%     \item Formularium    
%   \end{itemize}
% }

%---------- Richtlijnen -------------------------------------------------------
% Algemene richtlijnen worden automatisch ingevoegd (in hetzij Nederlands,
% hetzij Engels). Als er geen extra instructies nodig zijn, kan je dit leeg
% laten. Extra instructies kunnen hier aangevuld worden. Deze worden
% in een tabel (met één kolom) ingevoegd, dus eindig elke instructie met \\
% Laat elke regel ook voorafgaan door \midrule om de lijnen tussen de
% instructies correct in te voegen.

\Instructions{%
  \midrule
  \textbf{Kijk a.u.b.~je antwoorden grondig na voor indienen.} \\
  \midrule
  Als het antwoord op een vraag een reëel getal is, rond dat telkens af tot vier cijfers na de komma. Niet drie. Niet vijf. \textbf{Vier.} \\
}

\begin{questions}

\framedsolutions
\ifsolution
  \printanswers
\else
  \noprintanswers
  \newpage
\fi

%----------------------------------------------------------------------------
% Examenvragen
%----------------------------------------------------------------------------

\question[10] Waarom zijn de bananen krom?

\begin{solutionordottedlines}[2cm]
  Daarom!
\end{solutionordottedlines}

\question[1] Wie hoort niet thuis in het rijtje?

\begin{oneparcheckboxes}
  \choice George
  \choice John
  \choice Paul
  \choice Ringo
  \CorrectChoice Yoko
\end{oneparcheckboxes}

\question[1] Lorem ipsum dolor sit \fillin[amet][5cm]

\end{questions}

\ScratchNotes

\end{document}
